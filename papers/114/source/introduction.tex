\section{Introduction}
\label{lbl:intro}
%----------------------------------
Artificial Intelligence (AI) has permeated all aspects of our lives and has significantly transformed the way we live and work. It is therefore critical to prepare a talented workforce capable of ensuring the country’s competitiveness in this global economy. Computer science graduates are widely in demand, and part of that demand is that they are conversant with the skills and knowledge required to address cutting-edge problems. The recent explosion of AI and the increasing demand for AI professionals, make it imperative that our students be prepared for the workforce. In addition, while recent data from the Department of Labor projects a bright job outlook and a great demand for a computing workforce, many universities are still experiencing significant declines in first-to-second year retention of computer science majors~\cite{stephenson2018retention}.

While computing technology and applications have changed significantly over the last few decades, the most common approach for teaching introductory courses is still programming-focused~\cite{reiser2024s,kumar2024computer}.
%10.1145/3664191}. 
Despite efforts to change the perception, students still equate computer science with programming and many do not see its applicability. These disturbing retention figures combined with the need to prepare a workforce conversant with AI skills pose significant challenges and calls for a re-evaluation of our teaching models,  particularly in the introductory level courses. The need for transformations in undergraduate computing education has recently been recognized by several studies including some sponsored by NSF~\cite{chow2021teaching,hundhausen2022designing,raj2024ai,aly2024computer}. 
%-----------------------------------
%The Rebooting Computing Summit %\footnote{\url{http://www.rebootingcomputing.org/}} in January 2009 acknowledged that now is a time of challenges for the computing field and computing education needs to address those challenges.  

%-------------------------------------
%The proposed project
Our project, \textit{Infusing Artificial Intelligence Concepts into Introductory Computing Courses} (\textbf{Project InfuseAI}) intends to address the need to revitalize computer science curricula, enhance the learning experience of students in introductory computer science courses, and motivate further study in computing by (1) applying and connecting core Computational Thinking principles such as algorithmic reasoning, data representation, and computational efficiency to real-world challenges, with AI serving as the central theme,
%applying and relating fundamental Computational Thinking concepts of algorithmic thinking, data representation, and computational efficiency to real-world problems using AI as a theme, 
(2) implementing a set of hands-on laboratory projects in a wide range of applications using real-world problems which, when solved computationally, will illustrate both human problem-solving behavior and fundamental computer science concepts and (3) developing, applying, and testing an adaptable framework for the presentation of the above core concepts, allowing students to learn and apply core computing concepts to a broad range of societal challenges and opportunities.
%------------------------------------

While programming is an important skill, the introductory courses are often the first exposure students have to the field, and as such these courses should embrace the fundamentals of computer science. This includes the mathematical and logical background necessary for problem-solving using computers or computation in general. Teaching these fundamentals in computer science is now widely acknowledged as the most important component of teaching introductory computer science courses. The basic ideas behind this approach are nicely articulated ``as a way of solving problems, designing systems, and understanding human behavior that draws on concepts fundamental to computer science'' \cite{CMComthink}. This definition significantly broadens the classical goals of computing education and allows for new ideas that may help improve student experiences in introductory computing courses.
%------------------------------------------

Project InfuseAI builds on a successful NSF-funded project that explored the project-based approach to teaching the introductory artificial intelligence course using the unifying theme of Machine Learning (ML), and is an effort to create a model for revitalizing undergraduate computer science at the introductory computing level. We present work that involves the development of a model for the presentation of core computational topics based on a scaffolded approach to scale down the earlier developed AI-themed projects.  Our basic idea is to use interesting application-based real-world problems which, when solved computationally, will illustrate both human problem-solving behavior and fundamental computer science concepts. This approach is not new to computer science education, but it has usually been applied at later stages in the curriculum.  In such courses, interesting themes are used to unify the material. These themes are then presented in a project-based teaching environment so that they further enrich the student experiences with practical approaches and tools for problem-solving. Our previous NSF-funded experience is based on such an approach, where we incorporated ML as a unifying theme for introductory Artificial Intelligence courses through hands-on lab projects.  As a theme, ML provides methodology and technology to enhance real-world applications. 