\section{Evaluation Plan}
\label{lbl:evaluation}
%---------------------------------
%Limited evaluation with similar projects was done in the introductory courses in prior years with positive results~\citeanon{neller2008throw}. The two projects described above are being implemented during Spring 2025 and beyond at two institutions of higher education: a state university and a comprehensive private university. The effectiveness of the projects will be evaluated through an evaluation process involving both a formative evaluation and a summative evaluation. The formative evaluation will include strategies for monitoring the project as it evolves and will provide feedback to guide the development efforts of educational modules. An assessment of the educational modules' effectiveness in improving student learning and an evaluation of our efforts at different stages of the project will be used to assess progress made toward the project's goals. It will direct the development of the work. The summative evaluation will involve strategies to evaluate the effectiveness of the curricular material and the project achieving its goals. 

%--------------------------------------
%We are currently working on the development of an evaluation plan, including the generation of instruments, data collection techniques, and data analysis. Quantitative data will include pre- and post-instruments designed to gather in-depth information from students about the course work. In addition, quantitative data will also reflect students' perceptions of the field, changes in their understanding of AI, and any areas within the course work or process that they believe could be improved upon. The quantitative pre-post-instrument will include Likert-type scale items, and analysis of the data will include both descriptive and inferential statistics. Qualitative data will be gathered through focus groups, using a focus group protocol. 
%Qualitative data collected from these groups will be analyzed to identify recurring themes among stakeholders and will serve to complement and contextualize the quantitative findings.
%---------------------------------------
%%Qualitative data gathered through these groups will be analyzed for themes across stakeholders and used to support quantitative findings. 
%---------------------------------
%Since ongoing evaluation will be a systematic part of this project, adjustments to the curriculum take place in light of the evaluation results and as determined by feedback from students, and instructors teaching the courses. The results of these efforts will be disseminated in various venues. 
%======================================
%--------------------------------------
\begin{table}
\scriptsize
  \centering
  \caption{Survey Results during Spring 2025 } \label{tab:survey}
  \begin{tabularx}{\textwidth}{|X|l|}
    \hline
    The student project contributed to my overall understanding of the material in the course.
 & 86\% \\
    \hline
    The student project was at an appropriate level of difficulty given my knowledge of computer science and programming. & 100\% \\
    \hline
    After completing this project, I feel that I have a good understanding of the fundamental concepts covered in this course. & 75\% \\
    \hline
    The student project took a reasonable amount of time to complete. & 88\% \\
    \hline
    The student project was an effective way to introduce some basic artificial intelligence concepts. & 63\% \\
    \hline
    I have a firm grasp of the problem-solving techniques covered in this course. & 100\% \\
    \hline
    I would like the opportunity to apply some of the AI problem-solving techniques in the future. & 100\% \\
    \hline 
    I had a positive learning experience in this course. & 88\% \\
    \hline
    
  \end{tabularx}
\end{table}
%--------------------------------------
A preliminary evaluation of the two projects was conducted in the introductory courses, yielding positive results. Each of the projects has been implemented in CS1 or CS2 in at least one of the two. Most recently, and in order to evaluate students’ reception to our approach and to evaluate its effectiveness, we designed a short survey that was given in CS1 at the end of the spring 2025 semester. The survey includes several Likert-scale questions along with two qualitative questions. The Spring 2025 CS1 class consisted of 11 students, with a 73\% participation rate. 

%---------------------------------------
Below is a summary of the most recent survey results, conducted during spring 2025 using the Interactive Game project. The second column represents the percentage of students who agreed or strongly agreed. Overall, student responses were positive, as can be seen in Table 1. Similar results were reported in prior years [anonymous]. 
%---------------------------------------
Included in the short survey are two open-ended questions allowing students to provide feedback:

Q1: Describe what you liked best about the student project.

Q2: Describe what you liked least about the student project.

Students enjoyed the hands-on approach. They commented that the project was fun and a fair assignment based on what they had learned in the course. They enjoyed building the different parts of it. A student stated that while they struggled a little with some parts of the project, it was fun to complete. They enjoyed the opportunity to customize. 
The responses to Q2 centered around wishing to do more with AI and create more of the classes themselves.  
Based on the survey responses and informal interactions in class, the feedback was very encouraging.  While the sample data so far is small to draw significant conclusions, given these preliminary positive experiences and the feedback we received, the proposed project will allow us to develop more such programming projects and do a more comprehensive assessment at both institutions.
