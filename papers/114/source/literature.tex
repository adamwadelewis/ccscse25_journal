\section{Literature Review}
\label{lbl:literature}
%--------------------------------
Introducing AI in the introductory computer science courses takes two major approaches. The first one is based on using AI tools and techniques to enhance the student experiences in learning fundamental computer science concepts. Most of the research in this area is related to the recently gaining popularity of Large Language Models (LLM) and Generative AI. In~\cite{10.1145/3595634}, the authors discuss the recent changes in math education that emphasize teaching different representations of math problems that help students better understand the concepts behind them. Applying this approach in introductory computer science means exploring different representations of a programming problem, like higher-level text descriptions, flow charts, UML diagrams, or code. The idea is, instead of writing code to solve a problem given in another representation, to use AI-generated code that students explain or modify in higher-level terms, thus allowing them to better understand the solution and the programming concepts behind it. A similar approach is taken in ~\cite{10.1145/3626252.3630817}, however more focused on how LLMs work and on finding proper prompts to get the solution and then verifying and explaining it.  Students learning programming in CS1 are given a problem and tasked to design proper prompts for the AI system to generate code, which they verify. Thus, effective prompt engineering helps students better understand the solution to the problem and fundamental CS concepts such as software verification. In~\cite{10.1145/3626252.3630938,10.1145/3626253.3633433}, the authors developed an introductory CS course entirely based on the use of LLMs and other AI tools. It’s an online course where AI assists students with learning the course curriculum. These AI tools include LLMs that explain code and evaluate code style and a chatbot for answering course-related inquiries. In~\cite{10.1145/3649217.3653584}, the authors developed an introductory programming course called CS1-LLM that integrates AI tools into the course curriculum. Along with the basic CS material, the course teaches students the basic principles of AI tools like GitHub Copilot and LLMs, and how to use them to solve programming problems.  
%----------------------------------

The second approach to integrating AI in the introductory programming courses is based on introducing AI or ML problems into the course curriculum. The authors of~\cite{Bogaerts_2024} suggest an approach to teach CS1 topics in the context of solving AI problems. The exploration of AI is focused on weekly lab assignments and multi-week projects. The labs and projects task students to write programs that use AI and ML algorithms such as decision tree learning, matrix operations on datasets, and neural networks. They are provided with the code implementing these algorithms and incorporate it into their projects using the programming concepts they learn in CS1. In~\cite{10.1145/3408877.3432530}, the authors developed the curriculum of an introduction to computing course that may be taught to CS majors and non-majors. The course topics are explored by AI block lessons and a project. The lessons cover Introduction to AI, AI agents, and Introduction to ML, and in the project, students write a program to simulate a rocket landing using some AI techniques such as rule-based agents and genetic programming. For the AI components of their programs, students are provided with code, which they integrate using the basic programming concepts they learn in the introductory programming course. 
%---------------------------

Most of the existing approaches use AI and ML either to facilitate learning of the basic concepts in CS1/CS2 or to include AI and ML topics directly in the curriculum. 
Our approach differs in that it emphasizes fundamental computational thinking concepts such as algorithms, data representation, and computational efficiency, which are taught in CS1/CS2, by showing how they can be used to build AI and ML applications. In this way students can better understand these concepts and their importance for real-world computer science applications.
%=======================================