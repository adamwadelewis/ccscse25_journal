\documentclass{article}

\input{preamble}
\addbibresource{cfit.bib} 

\title{Broadening Participation in Computing Through Active Learning and Peer Mentorship: A Case Study of the C-FIT Freshmen Onboarding Program\footnote{\protect\input{copyright}}}

\author{
Eric Betties, Sofia Mata Avila, Aniyah Tucker,\\
Dale Marie Wilson, Marlon Mejias\\
College of Computing and Informatics\\
University of North Carolina at Charlotte\\
Charlotte, NC 28223\\
\email{\{ebetties,smataavi,atucke70,dwilso1,mmejias\}@charlotte.edu}
}

\begin{document}
\maketitle

\begin{abstract}
The College of Computing and Informatics Forty-Niner Intensive Transition (C-FIT) at the University of North Carolina at Charlotte (UNCC) is a freshman onboarding program designed to enhance success and foster a sense of belonging among new computing majors, particularly those with limited programming experience. C-FIT blends flipped classroom instruction, hands-on lab activities, and peer mentorship to build foundational computing skills and community. We present a mixed-method evaluation of three years of C-FIT, incorporating academic performance data, retention outcomes, and student feedback from surveys and focus groups. C-FIT participants earned higher grades in the introductory computer science course, remained in the major at higher rates, and reported increased confidence and a stronger sense of belonging compared to non-participants. These findings suggest that intensive early intervention can help bridge experience gaps and improve persistence among underrepresented computing students. We also discuss recommendations for scaling the program, refining the curriculum, and promoting broader adoption of the C-FIT model.
\end{abstract}

\section{Introduction}

The College of Computing and Informatics Forty-Niner Intensive Transition (C-FIT) at the University of North Carolina at Charlotte is a mini-fall bridge initiative designed to support the transition of incoming computing students into university life. The purpose of C-FIT is to build academic confidence, foster peer connections, and introduce students to campus resources prior to the formal start of their degree programs. C-FIT targets students who are entering an Introduction to Computer Science Principles course. The program has been running for three years and has welcomed students with varying levels of experience: from those with no programming background to self-taught learners looking to solidify foundational skills.

A faculty member leads the instruction, supported by a Ph.D. student and a team of teaching assistants who facilitate labs and hold office hours. Peer mentors (advanced undergraduates) are embedded throughout the program, maintaining a low student-to-mentor ratio (approximately 6:1) to encourage frequent interaction, support, and engagement. This high-contact model helps incoming students form connections with faculty and peers before the semester begins, addressing social and academic integration early.

Bridge programs in STEM are designed to address academic and social transition needs. They have been shown to increase students’ academic readiness, promote inclusion in the college community, and increase self-efficacy and persistence \cite{bradford2021,barth2021}. Studies show that participation in summer bridge initiatives can lead to a moderate increase in first-year GPA and significantly higher first-year retention rates for students \cite{bradford2021}. These programs often focus on serving minoritized students as a way to broaden participation in computing and other STEM fields \cite{washington2020}.

The C-FIT initiative aligns with these goals by combining early academic preparation with a strong support network. In particular, C-FIT emphasizes active learning strategies such as a flipped classroom model and hands-on labs. Active learning benefits all students and can narrow achievement gaps for underrepresented groups in STEM, leading to higher exam performance and lower failure rates \cite{theobald2020}. By embedding peer mentors and fostering a collaborative, inclusive learning environment, C-FIT also aims to improve students’ self-efficacy and sense of belonging, which are strongly linked to persistence in STEM majors \cite{barth2021,estrada2018}.

\section{Problem and Motivation}

Students with limited prior experience in computer science often face confirmation bias, believing that Computer Science 1 (CS1) courses are intended for more advanced learners. Their lack of experience, combined with the belief that CS1 is designed for advanced students, can negatively affect their performance \cite{tew2010}. Many of these students come from lower-income communities, including racially and ethnically minoritized groups (REM) \cite{washington2020}. 

At the time C-FIT was developed, UNCC offered the University Transition Opportunities Program (UTOP), which supported underrepresented students in their transition to college through general education coursework. Although UTOP included enrollment in CS1 (ITSC 1212), it lacked discipline-specific preparation to promote success in computing. As a result, students often entered CS1 without the foundational skills or confidence needed to succeed. 

To address this gap and counteract confirmation bias—the tendency among novice CS students to interpret introductory courses as geared toward those with prior experience—C-FIT was created as a targeted intervention. By providing tailored support and skill-building opportunities, C-FIT aims to enhance equity and confidence among students entering ITSC 1212.

\section{Background and Related Work}

Bridge and transition programs in STEM have demonstrated promising outcomes in improving student readiness, academic performance, and retention, particularly among underrepresented groups \cite{bradford2021}. Previous research \cite{kirkpatrick2017} has shown that students entering computer science programs with little or no previous programming experience tend to struggle in introductory CS courses, which can negatively impact retention in the major. Previous exposure to computing concepts significantly affects student grades and self-efficacy, and those with a programming background usually score higher marks in labs and exams \cite{kirkpatrick2017}.

To address such disparities, institutions have implemented summer bridge and early start programs that acclimate students to college-level expectations and foster early engagement with academic resources and peer support networks. Bradford et al. \cite{bradford2021} found that intensive transition initiatives offering mentorship, introductory coursework, and university orientation contributed to improved academic preparedness and social integration for first-generation and underrepresented students.

At UNCC, UTOP has long helped students adjust to college life; however, its curriculum does not focus on discipline-specific readiness. In contrast, the C-FIT framework allows colleges within the university to tailor bridge experiences for their disciplines. C-FIT was developed to provide foundational instruction in programming, mentorship, and campus navigation resources specific to computing majors. This paper expands on previous literature by examining the specific impact of C-FIT on readiness, persistence, and student perception of belonging.

\section{Approach and Uniqueness}

The C-FIT program is a five-week early fall bridge experience uniquely structured to blend academic instruction with peer mentoring and campus immersion. It launches roughly ten days before the start of the fall semester, with an intensive in-person week followed by four weeks of asynchronous online reinforcement. This hybrid format gives students a high-touch introduction to computing, as well as a flexible opportunity for concept mastery.

The curriculum is based on a flipped classroom model: students complete preparatory modules, short videos, and interactive texts prior to in-person sessions. During daily morning blocks (9:00 AM–12:00 PM), participants engage in hands-on exercises, mini-lectures, and active learning sessions led by a faculty instructor. The afternoons are devoted to structured lab activities and guided participation in university resource sessions (e.g., financial literacy workshops, campus tours, and academic support introductions). Each cohort of students moves through the program together to reinforce the identity and belonging of the group.

Peer mentors—selected upperclassmen and graduate students—play an integral role by maintaining a 1:6 mentor-to-student ratio. They provide guidance during lab activities, lead evening study halls (6:00–8:00 PM), and hold weekly office hours throughout the program. This socio-academic model improves both cognitive and affective outcomes by reducing barriers to support.

C-FIT also aligns with the College Center for Education Innovation and Research to incorporate active learning principles into early CS education. Its emphasis on pre-semester engagement, learning-by-doing, and sustained mentorship distinguishes C-FIT from traditional CS1 instruction or broader university-wide transition programs.

\section{Results}

During three years of implementation (2021, 2022, and 2024), C-FIT has produced substantial positive outcomes in the academic, psychosocial, and retention-related domains. These findings are based on institutional data, student surveys, course grades, and focus group feedback.

\textbf{Academic gains:} Across cohorts, the percentage of participants in C-FIT earning an A in the introductory CS course rose from 60\% in 2021 to 71\% in 2022, and then to 95\% in 2024. This trend suggests continuous improvement in the instructional design and support structures over time. The introduction of more structured labs, consistent peer mentoring, and targeted curricular adjustments each year likely contributed to these gains. In particular, the grade distributions of the C-FIT participants far exceeded those of the general CS1 student population during the same periods.

\begin{figure}[ht]
  \centering
  \includegraphics[width=0.85\linewidth]{grade_distribution.png}
  \caption{Grade distribution of C-FIT participants by cohort year (2021, 2022, 2024). Each bar indicates the percentage of students earning A, B, or C grades.}
  \label{fig:grade-distribution}
\end{figure}

\subsection*{Belonging and Self-Efficacy}

The sense of belonging among students in computing improved markedly after participating in C-FIT. Survey data showed a 45-point increase in the percentage of students who reported “feeling a sense of belonging in computing” (from 40\% before C-FIT to 85\% after), and a similar increase in those who felt they “had a voice in computing” (from 35\% to 82\%). The perception of having a supportive network in the computing program rose from 45\% to 90\%. These outcomes support the findings by Barth et al.~\cite{barth2021} that positive early orientation and peer connections can increase STEM self-efficacy and community integration.

\subsection*{Retention Indicators}

C-FIT participants have demonstrated strong continuation in the computing major. For example, 93\% of the 2021 cohort remained enrolled in a computing major one year after the program, with comparable retention rates observed for the 2022 and 2024 cohorts. In addition, C-FIT students earned higher CS1 course averages and had lower D/F/W (drop, fail, withdraw) rates than their peers who did not attend C-FIT.

\begin{table}[ht]
\centering
\begin{tabular}{|l|c|c|}
\hline
\textbf{Metric} & \textbf{Pre (\%)} & \textbf{Post (\%)} \\
\hline
Belonging & 40 & 85 \\
Voice in CS & 35 & 82 \\
Support Network & 45 & 90 \\
\hline
\end{tabular}
\caption{Pre- and post-program self-reported student perceptions of belonging in computing, voice in the discipline, and support network.}
\label{tab:survey}
\end{table}

These academic and psychosocial outcomes suggest that C-FIT functions as both an accelerator of foundational knowledge and a scaffold for identity formation in computing. In addition to these metrics, participant satisfaction with the program was exceptionally high. Nearly all C-FIT attendees reported feeling more connected to the campus community and better prepared for college coursework after completing the program. Furthermore, almost every participant indicated they would recommend C-FIT to future students. This feedback underscores the program’s success in fostering a supportive, inclusive environment for new computing majors.

\begin{figure}[ht]
  \centering
  \includegraphics[width=0.9\linewidth]{2024_FIT_survey_results.png}
  \caption{C-FIT 2024 survey results showing student-reported gains in belonging, voice, and support.}
  \label{fig:survey-2024}
\end{figure}

\section{Discussion}

The C-FIT bridge program provides a structured and replicable model of early intervention that builds academic confidence and social belonging among incoming computing students. As evidenced by performance trends and student feedback, intentional layering of flipped instruction, near-peer mentorship, and community integration yields strong results. Students not only achieve high grades in their first CS course, but also report significant gains in confidence and sense of belonging. These outcomes align with previous findings \cite{barth2021,theobald2020}, which emphasize the importance of active learning and early identity development in improving STEM persistence. Many participants specifically cited the supportive structure and mentor engagement as critical to their persistence during the early college transition.

Implementing C-FIT also revealed practical challenges and areas for improvement. Sustaining student motivation during the asynchronous portions of the program and effectively accommodating the wide range of incoming experience levels proved difficult. Future iterations could incorporate digital tracking tools, virtual peer accountability systems, and more structured online content delivery to maintain engagement after the in-person week. At the same time, key success factors include thorough instructor preparation and early alignment of mentors with the program’s pedagogical approach. Overall, these findings underscore that student participation in computing can be significantly enhanced through a highly personalized learning environment. This principle should guide future scaling efforts and aligns with evidence that robust institutional support is essential to increasing student persistence \cite{toven2015}.


\section{Conclusion}

C-FIT demonstrates that a short, intensive bridge program can meaningfully ease the transition into computer science. Participants consistently show strong academic performance, increased confidence, and a greater sense of belonging compared to their peers. These outcomes are especially promising for broadening participation in computing among students from historically underrepresented backgrounds. They also reinforce broader evidence for the efficacy of bridge programs \cite{bradford2021}, and highlight the value of sustained mentorship and active learning as core components of student success.

Looking ahead, the next phase of C-FIT will focus on longitudinal impact evaluation and scaling the model to reach more students. Planned enhancements include differentiated tracks to accommodate varying incoming skill levels, improved online scaffolding, real-time analytics to support remote engagement, and expanded partnerships with other institutions to disseminate the program. With early indicators of success—including high retention rates in CS1/CS2 and strong student satisfaction—C-FIT offers a model that can be adopted and adapted to promote equity and persistence in undergraduate computing education.

Through its intentional design and demonstrated results, C-FIT contributes to the national conversation on broadening participation in computing through strategic early interventions.


\printbibliography

\end{document}









