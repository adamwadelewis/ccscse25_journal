\documentclass{article}

\input{preamble}

% Title and Author
\title{Embracing the Ethical Use of Artificial Intelligence to Enhance Writing Skills and Comprehension\footnote{\protect\input{copyright}}}

\author{
Rahaf Barakat, Richard Nicklas, Lorraine Jonassen, David Kerven\\
School of Science and Technology\\
Georgia Gwinnett College\\
Lawrenceville, GA 30043\\
\email{\{rbarakat, rnicklas, ljonassen, dkerven\}@ggc.edu}\\
}


\begin{document}
\maketitle

\begin{abstract}
AI is rapidly transforming how students engage in education particularly, with writing and comprehension. The goal from this study is to examines student perceptions of AI’s usefulness in enhancing these skills. Conducted at Georgia Gwinnett College (GGC) in Fall 2024 and Spring 2025 in IT Professional Practice and Ethics course. Initially, students completed chapter analyses without AI, then with AI, and finally chose whether to continue using it. Most reported increased efficiency, improved grammar, and deeper comprehension of chapter topics. While AI assisted in drafting, students reported that they actively modified the output, maintaining accountability. Findings suggest that students leveraged AI as a tool to improve efficiency and comprehension while still applying critical thinking to their analyses and discussions.
\end{abstract}

\section{Introduction}

\subsection{The Evolution of Artificial Intelligence}

Artificial Intelligence (AI) has evolved from a theoretical concept into a transformative force across nearly every sector, including healthcare, business, and especially education. The foundations of AI were laid at the Dartmouth Conference in 1956, where McCarthy, Minsky, Rochester, and Shannon proposed the possibility of simulating intelligence through machines \cite{mccarthy1956proposal}. Over the decades, AI has progressed from rule-based systems to powerful machine learning, deep learning, and natural language processing (NLP) models, enabling advancements in tasks such as image classification, predictive analytics, and human-like text generation \cite{russell2020artificial, corpuz2025dreams, haenlein2019brief}.

\subsection{AI Integration in Education}

AI has become increasingly embedded in educational tools and platforms, redefining how students learn, write, and comprehend material. Applications like Grammarly, ChatGPT, Turnitin, and QuillBot now assist with grammar, idea generation, paraphrasing, and summarization \cite{garg2024aiwriting, willia2024ai}. Studies show that such tools can improve students' writing accuracy, fluency, and efficiency \cite{zawacki2019systematic, kim2024examining}. Adaptive learning systems powered by AI also personalize instruction, offering differentiated pathways based on individual learner data \cite{selwyn2024limits, jianzheng2023integration}.

\subsection{Student Perspectives and Benefits}

Student perceptions of AI play a critical role in determining its effectiveness and integration in the classroom. Research indicates that students appreciate AI for enhancing writing, comprehension, and task efficiency \cite{chan2023students, woerner2024transformative}. In a multi-institutional study, Johnston et al. \cite{johnston2024student} found that students valued AI tools for their ability to provide instant feedback, reduce cognitive load, and improve overall writing structure. Similar findings in STEM and humanities courses highlight AI’s perceived role in supporting autonomy and academic success \cite{woerner2024transformative, walter2024embracing}.

\subsection{Challenges and Ethical Considerations}

Despite these benefits, AI raises significant pedagogical and ethical concerns. Scholars warn that AI may promote surface-level learning, as students may prioritize efficiency over depth of comprehension \cite{selwyn2024limits}. Concerns have also been raised about academic integrity, authorship, plagiarism, and the blurring line between assistance and automation \cite{oye2024ethical}. Additional risks involve data privacy, algorithmic bias, lack of transparency, and overdependence on proprietary tools developed by commercial tech firms \cite{chan2023students, corpuz2025dreams, woerner2024transformative}.

\subsection{Research Gaps and Study Objectives}

Although the existing literature explores AI's technological capabilities and instructional applications, fewer studies focus on student-reported experiences and behavioral decisions, especially in writing-intensive college courses \cite{willia2024ai, chan2023students}. This study addresses this gap by investigating student perceptions of AI’s usefulness in writing and comprehension in the IT Professional Ethics and Practices course at Georgia Gwinnett College. Conducted in Fall 2024 and Spring 2025, this study followed a three-phase structure: —no AI use, required AI use, and optional continued use—within an upper-division IT ethics course required for all Information Technology concentrations. The course emphasizes professional responsibility, global IT ethics, and collaboration through student-led discussions and real-world case studies. The research aimed to examine whether students viewed AI as a tool for efficiency, a learning scaffold, or a barrier to critical thinking, and how these perceptions shaped their continued engagement with AI tools. Of 128 invited students, 111 completed the end-of-semester survey, yielding an 87\% response rate.

\section{Background}

Georgia Gwinnett College (GGC), founded in 2005 as part of the University System of Georgia, is a four-year public institution serving the Atlanta metropolitan area. With enrollment growing from 118 students in 2006 to nearly 12,000 by 2025, GGC offers over 27 majors across 6 schools. Designated as a Hispanic-Serving Institution (HSI) with over 25\% Hispanic/Latino students~\cite{ggcInstitutionalProfile}, it is among the most ethnically diverse colleges in the region.

\section{Methodology}

\subsection{Course Structure}
The course follows a student-led, discussion-based format integrating chapter analyses, peer discussions, and interactive activities to promote writing, critical thinking, and engagement. Each chapter analysis includes three sections: 1) summary, 2) key points, and 3) reflective opinion supported by reasoning and case studies. Class discussions encourage students to exchange perspectives and evaluate ethical issues in IT, while peer-led activities such as case studies, debates, and simulations provide experiential learning opportunities that connect theory to real-world ethical dilemmas.

\subsection{Method of Study}
Students were required to submit eight chapter analysis papers, with AI usage controlled in specific phases to assess its effects on efficiency, writing quality, and comprehension.

\subsection*{Phase 1: Baseline (No AI Usage)}
For the first two chapter analyses, students were instructed to complete their assignments without the use of any AI tools, including those for grammar correction or text generation.

\subsection*{Phase 2: AI-Assisted Writing}
For the next four chapter analyses, students were required to use AI tools of their choice, including those for grammar improvement (e.g., Grammarly) and/or text generation (e.g., ChatGPT). The grading criteria remained consistent with the previous phase. This phase aimed to assess how AI-assisted writing influenced efficiency, writing quality, and understanding of the course material, as well as learning about the array of AI tools students were using.

\subsection*{Phase 3: Student Choice}
For the final two chapter analyses, students were given the option to either continue using AI tools or return to writing their analyses independently. This phase examined whether they viewed AI as beneficial after initial use, offering insight into voluntary adoption for academic work.

After submitting the final analysis, students completed a 13-question anonymous survey capturing qualitative and quantitative data on AI usage (grammar, text generation, or both), perceived efficiency, impact on comprehension, changes in writing quality, and willingness to continue using AI.

\section{Results and discussion}
\textbf{AI Usage Patterns.} The first research question examined how students used AI—for grammar correction, text generation, or both. Combined results show that 61\% used AI for both purposes, 27\% for text generation only, and 12\% for grammar correction only, indicating that most students viewed AI as a comprehensive writing support tool.

\textbf{AI Impact on Efficiency.} The second research question explored whether AI improved efficiency in completing chapter analyses. Across both semesters, 47\% of students reported significantly less time spent, and 41\% reported less time overall. Only 12\% saw no improvement or increased time. Thus, nearly 88\% experienced greater efficiency, confirming AI’s value as a productivity aid, with minor delays linked to tool familiarity or content revision.

\textbf{AI Tools Used for Grammar Improvement and Text Generation} Research Questions 3 and 6 examined which AI tools students used to either improve grammar or generate content in their chapter analyses. Figure~\ref{paper_template/Figures/fig:ai_tool_usage} compares the top tools used across both purposes, based on combined data from Fall 2024 and Spring 2025.

This comparison reveals that while students often use the same tools for multiple functions, their frequency of use differs significantly depending on task. ChatGPT dominates in text generation, while Grammarly holds stronger for grammar refinement. The diversity of additional tools—including \textit{Perplexity}, \textit{Claude}, \textit{META}, and \textit{PowerPoint Summarizer}—suggests students are experimenting with different platforms to find the most suitable support for their writing needs.

\begin{figure}[h]
    \centering
    \includegraphics[width=0.7\linewidth]{paper_template/Figures/ai_tool_usage_comparison.png}
    \caption{Comparison of AI Tool Usage for Grammar Improvement vs. Text Generation (Fall 2024 \& Spring 2025 Combined)}
    \label{paper_template/Figures/fig:ai_tool_usage}
\end{figure}

\textbf{Impact of AI on Grammar, Writing Quality, and Topic Comprehension}  
The fourth research question examined whether students found AI helpful in improving grammar and writing quality. As shown in Figure~\ref{fig:ai_impact}, 19\% reported significant improvement, 46\% improvement, and 28\% slight improvement, while only 7\% saw no benefit. These results reflect GGC’s diverse student population, where AI likely supported language refinement and writing clarity for first-generation and non-native English speakers.

The fifth question explored AI’s influence on topic comprehension. Overall, 16\% reported significant improvement, 39\% general improvement, and 32\% no change, while 13\% noted decreased understanding. These findings suggest AI can enhance comprehension when used actively, though some students may rely on outputs without fully engaging with the material.


\begin{figure}[h]
  \centering
  \includegraphics[width=0.7\linewidth]{paper_template/Figures/ai_impact_split.png}
  \caption{Student-reported impact of AI on grammar, writing quality, and chapter comprehension.}
  \label{fig:ai_impact}
\end{figure}

\textbf{Impact of AI on Student Engagement with Learning Materials}  
Research Question 7 examined whether AI-generated text affected students’ reading habits. As shown in Table~\ref{tab:reading_habits}, 50\% reported spending less time reading, and 10\% skipped reading entirely. Meanwhile, 29\% maintained their usual reading time, and 11\% read more. These results indicate that while many students used AI to streamline work, a notable portion continued engaging with course materials. The 39\% who maintained or increased reading time suggest that AI does not completely replace active learning, underscoring the value of guided integration to promote meaningful engagement.

\begin{table}[h]
\caption{Student Reading Habits When Using AI to Generate Text} % title of Table
\label{tab:reading_habits} % reference label
\centering % used for centering table
\begin{tabular}{p{7.5cm} c} % two columns, first with fixed width
\hline\hline % double horizontal line
Behavior & Combined (\%) \\ [0.5ex] % table heading
\hline % single line
Spent Significantly More Time Reading the Chapter When Using AI & 4\% \\
Spent More Time Reading the Chapter When Using AI & 7\% \\
Read the Chapter the Same As When Not Using AI & 29\% \\
Spent Less Time Reading the Chapter When Using AI & 50\% \\
Did Not Read the Chapter at All When Using AI & 10\% \\ [1ex]
\hline % single line
\end{tabular}
\end{table}

\textbf{Student Modification of AI-Generated Text}  
Research Question 8 explored whether students revised AI-generated content. Most (78\%) modified the text before submission—19\% extensively and 59\% partially—while 22\% submitted it unchanged. These results indicate that most students critically engaged with the AI output, using it as a support tool rather than a substitute for their own ideas.


\textbf{Retention of Student Perspective in Personal Opinion Writing} Research Question 9 explored whether students retained their personal voice when using AI-generated text. Overall, 62\% wrote their own opinions, 21\% modified AI-generated ones, and only 17\% submitted unaltered content. These results indicate that most students critically engaged with AI, treating it as a writing aid rather than a substitute for personal reflection, while a small portion relied on it primarily for convenience.

\textbf{Impact of AI-Generated Text on Topic Comprehension}
Research Question 10 examined whether AI-generated text improved students’ understanding of course content. As shown in Figure~\ref{fig:ai_q10_q11}, 58\% reported improved comprehension (17\% significantly, 41\% moderately), 28\% did not see a change, and 15\% reported a decrease. These results suggest that AI can clarify complex ideas when used thoughtfully, though over-reliance may hinder deeper engagement.


\textbf{Impact of AI on Class Participation Preparedness} Research Question 11 explored how AI use affected students’ readiness for class discussions. As shown in Figure~\ref{fig:ai_q10_q11}, 51\% felt more prepared (13\% significantly), 27\% saw no change, and 22\% felt less prepared. These results suggest that AI can enhance confidence and conceptual readiness, though over-reliance may reduce genuine engagement.



\begin{figure}[h]
\centering
\includegraphics[width=0.95\linewidth]{paper_template/Figures/ai_q10_q11_split_horizontal.png}
\caption{Student-reported impact of AI use on comprehension of chapter topics (left) and preparedness for class participation (right).}
\label{fig:ai_q10_q11}
\end{figure}

\textbf{Student Choice to Continue Using AI} Research Question 12 explored whether students continued using AI when it was no longer required. Most (81\%) chose to do so, indicating clear perceived benefits in grammar, efficiency, and comprehension. This voluntary adoption suggests that students view AI as a lasting learning aid, similar to calculators in its integration into academic practice.


\textbf{Student Rationale for Continuing or Discontinuing AI Use}  
Research Question 13 examined students’ reasons for continuing or discontinuing AI use based on open-ended responses from Fall 2024. As shown in Tables~\ref{tab:reasons_continue} and~\ref{tab:reasons_discontinue}, most (80\%) continued using AI, citing efficiency, improved comprehension, ease of use, and writing support. Many used AI for initial drafts they later refined, noting reduced stress and benefits for learners with ADHD or limited English proficiency.


\begin{table}[h]
\caption{Reasons for Continuing AI Use (Fall 2024)} % title of Table
\label{tab:reasons_continue} % reference label
\centering % used for centering table
\begin{tabular}{p{3cm} p{1.4cm} p{6.5cm}} % adjusted widths
\hline\hline % double horizontal line
Theme & (\%) of Students & Representative Quote \\ [0.5ex] % table heading
\hline % single line
Efficiency & 33\% & ``It reduced the time needed for chapter analysis.'' \\
Comprehension & 21\% & ``It helped me understand the material better.'' \\
Ease of Use & 14\% & ``It was easier to submit assignments that way.'' \\
Writing Support & 12\% & ``It made my writing cleaner and more structured.'' \\
Grade Performance & 9\% & ``It helped me get good grades with less effort.'' \\
Accommodation & 5\% & ``I have a learning disability and it helped break content down.'' \\
Habitual Use & 4\% & ``I had already been using it all semester, so I continued.'' \\
Mixed Ethics & 2\% & ``Helpful, but I know it can also be misused.'' \\ [1ex]
\hline % single line
\end{tabular}
\end{table}

In contrast, the 20\% who discontinued AI use cited concerns about learning depth, preparedness, and ownership of ideas, preferring direct engagement with course content. Overall, these insights show that while students valued AI’s convenience, many also recognized its limits, emphasizing the need for guided and ethical integration in education.


\begin{table}[h]
\caption{Reasons for Discontinuing AI Use (Fall 2024)} % title of Table
\label{tab:reasons_discontinue} % reference label
\centering % used for centering table
\begin{tabular}{p{3cm} p{1.4cm} p{6.5cm}} % adjusted widths
\hline\hline % double horizontal line
Theme & (\%) of Students & Representative Quote \\ [0.5ex] % table heading
\hline % single line
Comprehension & 40\% & ``I felt like I didn’t understand the chapter... I wanted to do the work on my own.'' \\
Writing Independence & 20\% & ``I think that I can write and understand the information myself better.'' \\
Class Engagement & 15\% & ``It made me feel unprepared for the class discussions.'' \\
Self-Comparison & 15\% & ``I wanted to compare my writing with AI’s writing.'' \\
Instruction & 10\% & ``Professor said so, I think.'' \\ [1ex]
\hline % single line
\end{tabular}
\end{table}

\section{Conclusion and Future Work}

This study examined how IT students perceive and use AI tools to enhance writing and comprehension. Conducted in one upper-division IT ethics course, most students reported improved efficiency, understanding, and engagement. At Georgia Gwinnett College, AI particularly supported first-generation and non-native English speakers in developing writing clarity and structure.

Over 80\% of students revised or expanded AI-generated content, showing active learning rather than reliance. While results highlight the value of guided AI use, the study’s single-course scope and lack of control group limit generalization. Future research should include multiple courses, comparative groups, and structured AI instruction to assess long-term impacts on learning.


\medskip

\bibliographystyle{plain}
\bibliography{AI-ref}

\end{document}