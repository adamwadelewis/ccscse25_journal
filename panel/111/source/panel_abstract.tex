\documentclass{article}

\input{preamble}

\title{Navigating the effect of Generative AI on Undergraduate CS Majors \footnote{\protect\input{copyleft}}\\
\vspace{0.2in}
\large Panel Discussion
}

\author{
Adam Lewis\affmark[1], Laura Malave \affmark[2], Tania Roy\affmark[3], and Karen Works\affmark[4]\\
\affmark[1] Division of Mathematical, Computer, and Applied Sciences\\
College of Arts and Sciences \\
Athens State University, Athens, AL 35611\\
\email{Adam.Lewis@athens.edu}\\
\affmark[2]College of Computer and Information Technology\\
St. Petersburg College, St. Petersburg, FL 33733 \\
\email{malave.laura@spcollege.edu}\\
\affmark[3] Division of Natural Sciences\\
Computer Science \\
New College, Sarasota, FL 34243\\
\email{troy@ncf.edu}\\
\affmark[4] Computer Science \\
Florida State University, Panama City, FL 32405\\
\email{keworks@fsu.edu}
}

\begin{document}
\maketitle

\section{Summary}
Generative AI is a handy tool that facilitates software development. However, too high reliance on generative AI in introductory courses can cause students to miss key concepts of the programming languages \cite{bang2024impact}.

Learning to program is not easy. The method by which students circumvent the introduction to programming courses has evolved over time coinciding with developments in technology. In the 1980s, students copied code from other students in the same class or program \cite{shaw1980cheating}. With the development of the internet, blogs, and homework "helping" sites, students submitted existing programs from the world wide web and/or done by a third party, including hired and contracted \cite{manoharan2020contract,lancaster2021contract}. With the latest developments in generative AI, some students are utilizing AI generated solutions from systems such as ChatGPT \cite{ouh2023chatgpt}.
 
As opposed to the other methods, generative AI is becoming mainstream and integrated into development environments. The goal of institutes of higher learning is to graduate students who will have the skills to be successful in their careers. To reach these goals, the landscape of education will need to adapt to Chatgpt and similar AI technologies.

\section{Adam Lewis}
Athens State University is a 200-year-old institution that has been an upper division only university for more than fifty years. So, the needs of the adult learner and transfer student are paramount to our institution. The care and support of these students is very different from the usual population of 18-22
year old students that form the traditional body of college students recruited by our institutions.

The integration of generative AI tools into the software development process raises painful issues for computer science instruction.  We face the ethical issues of students using AI tools to generate solutions to assignments while claiming those generated solutions as their own work.  But there is a deeper issue regarding the instructional materials used in courses as the Large Language Models are trained on these materials.  Thus, the pedagogical techniques used in computer science instruction must evolve to a more active model of instruction.

It is said that from chaos arises opportunity.  As industry incorporates more of the generative AI tools into the software development life cycle, higher education can incorporate these tools into instructional techniques.   For example, adding AI tools into the process of pair programming can form a method for a more active instructional process to teach beginning programming.  We feel our connection at Athens State University with a non-traditional student population provides opportunity for improvement of how we teach computer science with these tools.

\section{Laura Malave}
St. Petersburg College’s College of Computer and Information Technology offers a dynamic and workforce-driven pathway for students pursuing careers in cybersecurity, programming, networking, data science, and related fields. Fully accredited and aligned with industry standards, CCIT provides stackable credentials—from certificates to A.S. and B.A.S. degrees—that support both traditional and non-traditional students, including working professionals. With flexible online and in-person options, hands-on labs, and strong partnerships with industry and government, the college emphasizes applied learning and career readiness. Students benefit from experienced faculty, internship opportunities, and pathways from Florida A.A. degrees. Graduates leave prepared to enter or advance in high-demand technology sectors with practical skills, certifications, and a solid academic foundation.

\section{Tania Roy}
New College of Florida, Florida's public honors liberal arts college, offers a Computer Science program with a rigorous and flexible curriculum rooted in the liberal arts tradition. Our SACS-accredited BA degree supports both traditional and non-traditional pathways, with opportunities for interdisciplinary exploration and individualized study. The program welcomes students who have completed a Florida A.A. degree. Through independent study projects, project-based courses, close faculty mentorship, and narrative-style evaluations, students are empowered to take ownership of their learning. Our unique curricular approach equips New College graduates with the critical thinking, technical expertise, and collaborative skills needed to succeed in a fast-paced, interdisciplinary world.

\section{Karen Works}
The Florida State University (FSU) online Computer Science program supports an ABET accredited BS degree and two SACS accredited BA degrees in Computer Science. Our programs accept students who have completed a minimum of 52 hours of credit at FSU, or an A.A degree. We serve a diverse non-traditional student population and support students around the world.

We are dedicated to ensuring that all FSU graduates regardless of whether they take the traditional face-to-face or nontraditional online route are prepared for demanding fast paced technology jobs. Towards this goal, FSU has developed many resources to support high quality online learning.

\section{Biographies}
\textbf{Adam Wade Lewis} is a Professor of Computer Science and Program Coordinator for Computer Science and Information Technology at Athens State University, in Athens, Alabama. They are actively involved with transfer and the academic advising process working with both community college transfer students and transfer students from other senior institutions.

\textbf{Laura Malave} is an Associate Professor of Computer and Information at St. Petersburg College, in St. Petersburg, Florida. She serves as the NSA CAE (Center of Academic Excellence in Cybersecurity) POC(Point of Contact). She serves as a board member of the ACM2Y, and the NCL (National Cyber League). 

\textbf{Tania Roy} is an Associate Professor of Human-Centered Computing at New College of Florida, where she is actively involved in curricular development and advising both traditional and non-traditional students pursuing majors and minors in Computer Science and Data Science.

\textbf{Karen Works} is an Associate Teaching Professor of Computer Science in the Computer Science Department on the Panama City campus of Florida State University. She is passionately involved in developing well designed instructional materials that promote active learning.

\medskip

\bibliographystyle{plain}
\small 
\bibliography{tutorial_abstract}

\end{document}
