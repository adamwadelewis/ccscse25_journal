\documentclass{article}

\input{preamble}

\title{Agentic AI and the Cyber Arms Race\footnote{\protect\input{copyleft}}\\
\vspace{0.2in}
\large{
Keynote
}}

\author{Sean Oesch, Senior Scientist of Oak Ridge National Laboratories}

\begin{document}
\maketitle

In the early years of cybersecurity, defenders utilized virus-specific
signatures, honeypots, and heuristics. As attacks increased in volume
and attackers became more sophisticated, moving toward polymorphic
malware, packers, and novel evasion techniques, defenders looked to
machine learning to provide scalability (quickly analyze large volumes
of data and automate repetitive tasks), pattern recognition (detect
common attack patterns), and novelty detection (recognize abnormal
behaviors that may indicate malicious actors or insider threats). With
the advent of deep learning-based reinforcement learning algorithms and
large language models we are on the cusp of another revolution in
cybersecurity - agentic artificial intelligence. In this talk, Sean
Oesch, a cyber researcher and senior scientist at Oak Ridge National
Laboratory, will discuss the implications of agentic AI for cyber
warfare and share his thoughts on how to educate AI savvy students who
can defend the networks and critical infrastructure of the future.


\end{document}

