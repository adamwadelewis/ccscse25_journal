\documentclass{article}

\input{preamble}

\title{Enhancing CS1 Engagement and Outcomes with Amplify-Supported POGIL Activities \footnote{\protect\input{copyleft}}
\\
\vspace{0.2in}
\large{
%A Nifty Assignment
% Poster Session
 Workshop
% Panel
}}


% Target typesetting:
%
%  Baochuan Lu, Author A, John Meinke, Author B
%        Computer and Information Sciences
%          Southwest Baptist University
%               Bolivar, MO 65613
%            {blu,author}@sbuniv.edu
%          Computer Science Department
%              Another University
%              Our Town, TX 00000
%           {jmeinke,author}@univ.edu

%\author{
%Baochuan Lu\affmark[1], Author A\affmark[1], John %Meinke\affmark[2], Author B\affmark[2]\\
%\affmark[1]Computer and Information Sciences\\
%Southwest Baptist University\\
%Bolivar, MO 65613\\
%\email{\{blu,author\}@sbuniv.edu}\\
%\affmark[2]Computer Science Department\\
%Another University\\
%%Our Town, TX 00000\\
%\email{\{jmeinke,author\}@univ.edu}
%}
\author{
Xin Xu, Evelyn Brannock, Wei Jin, Hyesung Park, Tacksoo Im\\
Department of Information Technology \\
Georgia Gwinnett College\\
1000 University Center Lane, Lawrenceville, GA 30024\\
\email{\{ebrannoc, xxu, wjin, hpark7, tim\}@ggc.edu}
}
\begin{document}

\maketitle
\section*{Workshop Description}
The introductory programming course—commonly referred to as CS1—is well known for its difficulty and high failure rates. Over the years, numerous instructional strategies have been proposed and researched to improve learning outcomes in CS1. One such approach is {\bf Process Oriented Guided Inquiry Learning (POGIL)} \cite{hu2014teaching, POGIL}.  


At our institution, some faculty have modified the document based POGIL activities originally developed by Hu et. al \cite{hu2014teaching}, and have adopted them for several years. In this model, students work collaboratively in structured teams to explore a sequence of carefully crafted scenarios and questions, constructing their own understanding of new programming concepts. Since its inception, research has shown that POGIL leads to decreased failure and withdrawal rates and increased numbers of A's and B's \cite{beneteau2016peer, farrell1999guided, ruder2008pogil} in courses in different disciplines.  


However, we encountered several practical challenges that emerged with this traditional implementation in CS1 coursework. The activities were time-intensive for students to complete during scheduled class periods. In addition, team management for the instructor could be difficult, particularly in addressing disengaged and online students. Third, instructors often struggle to monitor team progress for timely intervention. 


To address these issues, a group of faculty in our institution collaborated in Summer 2024 to redesign our POGIL activities using the {\bf Amplify} platform \cite{Amplify}. Amplify is a free, web-based tool that supports the creation of interactive, slide-based lessons. It offers powerful classroom management features that allow instructors to track student and team progress in real time, quickly identify misconceptions, and provide immediate feedback—resulting in a more adaptive and efficient learning experience.  


We piloted Amplify-supported POGIL activities in multiple CS1 sections during Fall 2024 and Spring 2025, involving five instructors teaching both control and treatment groups across in-person and online formats. Pre- and post-survey results will be presented at the workshop. Although response rates vary, preliminary high-level findings indicate strong positive sentiment: 
\begin{itemize}
    \item Amplify Learning: 45\% rated the experience as more than moderately effective; 17\% rated it less than moderately effective. 
    \item Amplify Motivation: 44\% found it more than moderately effective; 25\% rated it less than moderately effective. 
    \item Amplify Peer Interaction: 54\% rated it more than moderately effective; 23\% rated it less than moderately effective. 
\end{itemize}



These early results suggest the effectiveness of Amplify-supported POGIL activities in promoting both student learning and motivation. 


\section*{Workshop Agenda(90 minutes)}
\begin{itemize}
    \item Introduction to the POGIL Instructional Model (15 min) 

    \item Overview of Amplify as a Delivery Platform for POGIL (15 min) 

    \item Hands-on Creation of Amplify-Based POGIL Activities for CS1 (45 min) 

    \item Demonstration of Features for Monitoring and Supporting Student Engagement in Amplify (15 min) 
\end{itemize}

Access to a curated subset of Amplify-integrated POGIL activities will be provided to participants. Contact information will be collected at the workshop to facilitate distribution upon successful completion of the two-year project.  


\bibliographystyle{plain}
\bibliography{sample}

\end{document}
